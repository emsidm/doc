\documentclass[bulgarian,a4paper,12pt,titlepage]{article}

\usepackage[T2A,T1]{fontenc}
\usepackage[utf8]{inputenc}
\usepackage[bulgarian]{babel}
\usepackage{fullpage}
\usepackage{indentfirst}


%\usepackage{mathptmx} % Times New Roman

\inputencoding{utf8}
\linespread{1.25} % Word - 1.5

\setcounter{tocdepth}{2}

\title{%
Проект: EMS \\
\large Направление: Уеб Приложения}

\author{%
\textbf{Автор:} \\
Алберт Александров Стефанов \\
11 ``и'' клас, НПМГ \\
albertas@students.npmg.org \\
\and
\textbf{Ръководител:} \\
Мирослава Николова \\
Учител по ИТ, НПМГ \\
miroslava.nikolova@npmg.org}

\date{}

\begin{document}
    \maketitle
    \newpage
    \tableofcontents
    \newpage
    
    \section{Описание на проекта}
        EMS е система за управление на потребителските идентификатори (Identity Management). Идеята е породена от въвеждането на иновативни технологии в училищата --- системи за електронно обучение, ученически електронни пощи, безжични мрежи с персонални акаунти, електронни дневници и други. \par
        EMS предлага решение на този проблем. Чрез лесен за използване уеб-базиран панел, отговорните лица --- класни ръководители, ЗАТС, директор и заместник-директори, могат да създават потребители във всички системи в зависимост от ролята им (приемерно ученик или учител). Всяка позиция идва с ограничения за създаваните видове потребители. \par
        След въвеждане на информацията за даден човек в отделна система, файл или ръчно в EMS, системата създава потребителя спрямо зададения шаблон и го провизира в отделните системи, зададени първоначално от администратор.
    
    \section{Цели и целеви групи}
        \subsection{Цели}
            \begin{itemize}
                \item Интеграция между различни системи
                \item Автоматизация на провизирането и синхронизацията
                \item Намаляване на човешките грешки при провизиране
            \end{itemize}

        \subsection{Целеви групи}
            \begin{itemize}
                \item Училища и универститети
                \item Организации с голям брой системи и бази данни
                \item Корпорации с голям брой служители
            \end{itemize}
    
    \section{Основни етапи}
        \subsection{Проучване}
            \begin{enumerate}
                \item Проучване на конкурентните продукти
                \item Изисквани функции
                \item Технологии за имплементацията
            \end{enumerate}

        \subsection{Планиране}
            \begin{enumerate}
                \item Приоритизиране на функционалностите
                \item Първи системи за изграждане на интеграции
                \item Създаване на абстракциите
            \end{enumerate}
            
        \subsection{Имплементация}
            \begin{enumerate}
                \item Създаване на слоя за връзка с бази данни
                \item Имплементиране на синхронизатор
                \item Създаване на административен панел
            \end{enumerate}
            
    \section{Ниво на сложност}
        \subsection{Възникнали проблеми}
            \begin{itemize}
                \item Липса на cross-platform LDAP библиотеки с пълни функционалности --- паралелно с проекта се разбработва библиотека на име SimpleLdap
                \item Голяма разлика в необходимите функции на различните организации
            \end{itemize}

        \subsection{Разлики в необходимите функции}
            \begin{itemize}
                \item Различни конвенции за потребителските имена и електронните пощи --- задават се в конфигурацията
                \item Различни бази данни --- създадена е абстракция над конекторите с цел лесна подменяемост
                \item Необходимите бази данни трябва да се зареждат според конфигурация --- имплементацията наподобява plugin система
                \item Вътрешни разработки на организациите --- лесно може да се напише конектор за връзка с вътрешните разработки или да се работи със сурови данни от базата им
            \end{itemize}

    \section{Логическо и функционално описание}
        \subsection{Презентационен слой}
            \subsubsection{Потребителски панел}
                \begin{itemize}
                    \item Управление на потребители
                    \item Управление на потребителски роли
                    \item Създаване и управление на заявки за одобрение на права
                \end{itemize}
            \subsubsection{Административен панел}
                \begin{itemize}
                    \item Създаване на шаблони за потребители
                    \item Дефиниране на технически изисквания към бизнес роли
                \end{itemize}

            \subsubsection{Web API}
                \begin{itemize}
                    \item Извеждане на списък с потребители с цел интеграция в корпоративни сайтове
                    \item Извличане на статистики за репортинг системи на трети лица
                \end{itemize}

            \subsubsection{Email комуникация}
                \begin{itemize}
                    \item Изпращане на молби за одобрение по e-mail
                    \item Одобряване или отхвърляне чрез отговор на писмо
                    \item Не е необходимо служителят да е в корпоративната мрежа, за да управлява заявките за права
                \end{itemize}

            
        \subsection{Бизнес слой}
            \subsubsection{Синхронизатор}
                \begin{itemize}
                    \item Изгражда връзките между различните бази данни и системи
                    \item При подаване на команда на системата, провизира промените
                    \item Следи за ръчна намеса в промяната на правата и прави репорти за несъответствие на данните между целевите бази
                \end{itemize}

            \subsubsection{Мениджър за управление на права}
                \begin{itemize}
                    \item Превръща бизнес ролите в конкретни за всяка система права
                    \item Проследява заявките за одобрение на права
                    \item Дава заявка на синхронизатора да приложи промените във всяка система
                \end{itemize}
        \subsection{Слой за данни}
            \subsubsection{Системи - източници}
                \begin{itemize}
                    \item Съдържат въведените данни за потребителите
                    \item Използват се като източник на информация за бизнес ролите и личните данни
                \end{itemize}
            \subsubsection{Целеви системи}
                \begin{itemize}
                    \item Съхраняват техническите роли и ги прилагат
                    \item Получават единствено необходимите за функционирането им данни
                \end{itemize}
            \subsubsection{Смесени системи}
                \begin{itemize}
                    \item Едновременно са източници и целеви системи
                    \item Може да се използват когато е необходимо обратно да се записват данни, генерирани от други системи
                \end{itemize}
    \section{Реализация}
    \subsection{Презентационен слой}
            \subsubsection{Потребителски и административен панел}
                \begin{itemize}
                    \item Single Page Application, ReactJS
                    \item Responsive Design
                    \item Комуникират с Web API
                \end{itemize}

            \subsubsection{Web API}
                \begin{itemize}
                    \item .NET Core Web API
                    \item Предлага RESTful методи, даващи възможност за комуникация със системата
                    \item Конектор --- смесена система
                \end{itemize}

            \subsubsection{Email комуникация}
                \begin{itemize}
                    \item Email клиент --- приема и изпраща писма
                    \item Email генератор --- генерира писма, включващи заявките за права
                    \item Конектор --- система-източник
                \end{itemize}

            
        \subsection{Бизнес слой}
            \subsubsection{Синхронизатор}
                \begin{itemize}
                    \item Обработва заявките, насочени към определени системи
                \end{itemize}

            \subsubsection{Мениджър за управление на права}
                \begin{itemize}
                    \item 
                \end{itemize}
        \subsection{Слой за данни}
            \subsubsection{Системи - източници}
                \begin{itemize}
                    \item Имплементират интерфейс IDataSource
                    \item Използват се като източник на информация за бизнес ролите и личните данни
                \end{itemize}
            \subsubsection{Целеви системи}
                \begin{itemize}
                    \item Имплементират интерфейс IDataTarget
                    \item Получават единствено необходимите за функционирането им данни
                \end{itemize}
            \subsubsection{Смесени системи}
                \begin{itemize}
                    \item Имплементират интерфейс IDataContext, който включва възможностите както на IDataSource, така и на IDataTarget
                    \item Може и да се четат, и да се записват данни
                \end{itemize}
\end{document}
